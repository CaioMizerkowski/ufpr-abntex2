% ----------------------------------------------------------
% ELEMENTOS PRÉ-TEXTUAIS
% ----------------------------------------------------------
 %\pretextual

% ---
% Capa
% ---
\imprimircapa
% ---

% ---
% Folha de rosto
% (o * indica que haverá a ficha bibliográfica)
% ---
\imprimirfolhaderosto*
% ---

% ---
% Inserir a ficha bibliografica
% ---

% Isto é um exemplo de Ficha Catalográfica, ou ``Dados internacionais de
% catalogação-na-publicação''. Você pode utilizar este modelo como referência. 
% Porém, provavelmente a biblioteca da sua universidade lhe fornecerá um PDF
% com a ficha catalográfica definitiva após a defesa do trabalho. Quando estiver
% com o documento, salve-o como PDF no diretório do seu projeto e substitua todo
% o conteúdo de implementação deste arquivo pelo comando abaixo:
%
% \begin{fichacatalografica}
%     \includepdf{fig_ficha_catalografica.pdf}
% \end{fichacatalografica}
\begin{fichacatalografica}%\color{blue}
	\vspace*{\fill}					% Posição vertical
	\hrule							% Linha horizontal
	\begin{center}					% Minipage Centralizado
	\begin{minipage}[c]{12.5cm}		% Largura
	\FichaCatalografica
	
	\end{minipage}
	\end{center}
	\hrule
\end{fichacatalografica}
% ---

% ---
% Inserir errata
% ---
\begin{errata}%\color{blue}
Elemento opcional da \citeonline[4.2.1.2]{NBR14724:2011}. Exemplo:

\vspace{\onelineskip}

FERRIGNO, C. R. A. \textbf{Tratamento de neoplasias ósseas apendiculares com
reimplantação de enxerto ósseo autólogo autoclavado associado ao plasma
rico em plaquetas}: estudo crítico na cirurgia de preservação de membro em
cães. 2011. 128 f. Tese (Livre-Docência) - Faculdade de Medicina Veterinária e
Zootecnia, Universidade de São Paulo, São Paulo, 2011.

\begin{table}[htb]
\center
\footnotesize
\begin{tabular}{|p{1.4cm}|p{1cm}|p{3cm}|p{3cm}|}
  \hline
   \textbf{Folha} & \textbf{Linha}  & \textbf{Onde se lê}  & \textbf{Leia-se}  \\
    \hline
    1 & 10 & auto-conclavo & autoconclavo\\
   \hline
\end{tabular}
\end{table}

\end{errata}
% ---

% ---
% Inserir folha de aprovação
% ---

% Isto é um exemplo de Folha de aprovação, elemento obrigatório da NBR
% 14724/2011 (seção 4.2.1.3). Você pode utilizar este modelo até a aprovação
% do trabalho. Após isso, substitua todo o conteúdo deste arquivo por uma
% imagem da página assinada pela banca com o comando abaixo:
%
% \includepdf{folhadeaprovacao_final.pdf}
%
\begin{folhadeaprovacao}%\color{blue}

  \begin{center}
    {\ABNTEXchapterfont
    {\bfseries\folhadeaprovacaoname}\par\phantom{}\par
    %\large
    \MakeUppercase\imprimirautor}

    \vspace*{\fill}\vspace*{\fill}
    \begin{center}
      \ABNTEXchapterfont
      %\bfseries\Large
      \MakeUppercase\imprimirtitulo
    \end{center}
    \vspace*{\fill}
    \begin{minipage}{\textwidth}
    \hspace{.25\textwidth}
    %\begin{minipage}{.5\textwidth}
        \imprimirpreambulo, pela seguinte banca examinadora:
    %\end{minipage}%
    \end{minipage}


    \vspace*{\fill}
   \end{center}
       \assinatura{\textbf{\imprimirorientador} \\ Orientador} 
   \AssinaAprovacao
  
\end{folhadeaprovacao}
% ---

% ---
% Dedicatória
% ---
\begin{dedicatoria}
   \vspace*{\fill}
   \centering
   \noindent
   \DedicatoriaTexto
    \vspace*{\fill}
\end{dedicatoria}
% ---

% ---
% Agradecimentos
% ---
\begin{agradecimentos}
\AgradecimentosTexto
\end{agradecimentos}
% ---

% ---
% Epígrafe
% ---
\begin{epigrafe}
    \vspace*{\fill}
	\begin{flushright}
        \EpigrafeTexto
	\end{flushright}
\end{epigrafe}
% ---

% ---
% RESUMOS
% ---

% resumo em português
\setlength{\absparsep}{18pt} % ajusta o espaçamento dos parágrafos do resumo
\begin{resumo}
    \ResumoTexto
    \vspace{\onelineskip}
    \noindent 
    \textbf{Palavras-chaves}: \PalavraschaveTexto
\end{resumo}

% resumo em inglês
\begin{resumo}[ABSTRACT]
 \begin{otherlanguage*}{english}
   \AbstractTexto
   \vspace{\onelineskip}
 
   \noindent 
   \textbf{Key-words}: \KeywordsTexto
 \end{otherlanguage*}
\end{resumo}

% resumo em francês 
% \begin{resumo}[RESUME]%Résumé
%  \begin{otherlanguage*}{french}
%     Il s'agit d'un résumé en français.
%  
%    \textbf{Mots-clés}: latex. abntex. publication de textes.
%  \end{otherlanguage*}
% \end{resumo}
% 
% % resumo em espanhol
% \begin{resumo}[RESUMEN]
%  \begin{otherlanguage*}{spanish}
%    Este es el resumen en español.
%   
%    \textbf{Palabras clave}: latex. abntex. publicación de textos.
%  \end{otherlanguage*}
% \end{resumo}
% ---

% ---
% inserir lista de ilustrações
% ---
\pdfbookmark[0]{\listfigurename}{lof}
\listoffigures*
\cleardoublepage
% ---

% ---
% inserir lista de tabelas
% ---
%\pdfbookmark[0]{\listtablename}{lot}
\listoftables*
%\cleardoublepage
% ---

% ---
% inserir lista de abreviaturas e siglas
% ---
\listarsiglas
\lsiglas  % listar siglas do arquivo 00-dados.tex
%\cleardoublepage
% ---

% ---
% inserir lista de símbolos
% ---
\listarsimbolos
 \lsimbolos % listar simoblos do arquivo 00-dados.tex
%\cleardoublepage
% ---

% ---
% inserir o sumario
% ---
%\pdfbookmark[0]{\contentsname}{toc}
\tableofcontents*
\cleardoublepage
% ---
