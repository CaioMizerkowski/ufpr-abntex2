%%%%%%%%%%%%%%%%%%%%%%%%%%%%%%%%%%%%%%%%%%%%%%%%%%%%%%%
% Arquivo para entrada de dados para a parte pré textual
%%%%%%%%%%%%%%%%%%%%%%%%%%%%%%%%%%%%%%%%%%%%%%%%%%%%%%%
% 
% Basta digitar as informações indicadas, no formato 
% apresentado.
%
%%%%%%%
% Os dados solicitados são, na ordem:
%
% tipo do trabalho
% componentes do trabalho 
% título do trabalho
% nome do autor
% local 
% data (ano com 4 dígitos)
% orientador(a)
% coorientador(a)(as)(es)
% arquivo com dados bibliográficos
% instituição
% setor
% programa de pós graduação
% curso
% preambulo
% data defesa
% CDU
% errata
% assinaturas - termo de aprovação
% resumos & palavras chave
% agradecimentos
% dedicatória
% epígrafe


% Informações de dados para CAPA e FOLHA DE ROSTO
%----------------------------------------------------------------------------- 
\tipotrabalho{Trabalho Acadêmico}
%    {Relatório Técnico}
%    {Dissertação}
%    {Tese}
%    {Monografia}

% Marcar Sim para as partes que irão compor o documento pdf
%----------------------------------------------------------------------------- 
 \providecommand{\terCapa}{Sim}
 \providecommand{\terFolhaRosto}{Sim}
 \providecommand{\terTermoAprovacao}{Sim}
 \providecommand{\terDedicatoria}{Sim}
 \providecommand{\terFichaCatalografica}{Nao}
 \providecommand{\terEpigrafe}{Sim}
 \providecommand{\terAgradecimentos}{Sim}
 \providecommand{\terErrata}{Nao}
 \providecommand{\terListaFiguras}{Sim}
 \providecommand{\terListaQuadros}{Nao}
 \providecommand{\terListaTabelas}{Sim}
 \providecommand{\terSiglasAbrev}{Nao}
 \providecommand{\terResumos}{Sim}
 \providecommand{\terSumario}{Sim}
 \providecommand{\terAnexo}{Nao}
 \providecommand{\terApendice}{Nao}
 \providecommand{\terIndiceR}{Nao}
%----------------------------------------------------------------------------- 

\titulo{Problema inverso aplicado à modelagem e linearização de transmissores sem fio}
\autor{Caio Phillipe Mizerkowski}
\local{Curitiba}
\data{2023} %Apenas ano 4 dígitos

% Orientador ou Orientadora
\orientador{Prof. Dr. Eduardo Gonçalves de Lima}
%Prof Emílio Eiji Kavamura, MSc}
%\orientadora{}
%Prof\textordfeminine~Grace Kelly, DSc}
% Pode haver apenas uma orientadora ou um orientador
% Se houver os dois prevalece o feminino.

% Em termos de coorientação, podem haver até quatro neste modelo
% Sendo 2 mulhere e 2 homens.
% Coorientador ou Coorientadora
%\coorientador{}%Prof Morgan Freeman, DSc}
%\coorientadora{Prof\textordfeminine~Audrey Hepburn, DEng}

% Segundo Coorientador ou Segunda Coorientadora
%\scoorientador{}
%Prof Jack Nicholson, DEng}
\scoorientadora{}
%Prof\textordfeminine~Ingrid Bergman, DEng}
% ----------------------------------------------------------
\addbibresource{referencias.bib}

% ----------------------------------------------------------
\instituicao{Universidade Federal do Paraná}

\def \ImprimirSetor{}%Setor de Tecnologia}

\def \ImprimirProgramaPos{}%Departamento de Engenharia Elétrica}

\def \ImprimirCurso{}%Curso de Engenharia Elétrica}

\preambulo{Primeira parte do Trabalho de Conclusão de Curso apresentado ao Curso de Graduação em Engenharia Elétrica, Departamento de Engenharia Elétrica, Setor de Tecnologia, Universidade Federal do Paraná, como requisito à obtenção do título de Bacharel em Engenharia Elétrica}
%do grau de Bacharel em Expressão Gráfica no curso de Expressão Gráfica, Setor de Exatas da Universidade Federal do Paraná}

%----------------------------------------------------------------------------- 

\newcommand{\imprimirCurso}{}
%Programa de P\'os Gradua\c{c}\~ao em Engenharia da Constru\c{c}\~ao Civil}

\newcommand{\imprimirDataDefesa}{22 de Setembro de 2022}

\newcommand{\imprimircdu}{}

% ----------------------------------------------------------
\newcommand{\imprimirerrata}{}

% Comandos de dados - Data da apresentação
\providecommand{\imprimirdataapresentacaoRotulo}{}
\providecommand{\imprimirdataapresentacao}{}
\newcommand{\dataapresentacao}[2][\dataapresentacaoname]{\renewcommand{\dataapresentacao}{#2}}

% Comandos de dados - Nome do Curso
\providecommand{\imprimirnomedocursoRotulo}{}
\providecommand{\imprimirnomedocurso}{}
\newcommand{\nomedocurso}[2][\nomedocursoname]
  {\renewcommand{\imprimirnomedocursoRotulo}{#1}
\renewcommand{\imprimirnomedocurso}{#2}}


% ----------------------------------------------------------
\newcommand{\AssinaAprovacao}{

\assinatura{%\textbf
   {Professor} \\ Prof. Dr. Marcelo de Souza}
   \assinatura{%\textbf
   {Professor} \\ Prof. Dr. Bernardo Rego Barros de Almeida Leite}
      
   \begin{center}
    \vspace*{0.5cm}
    %{\large\imprimirlocal}
    %\par
    %{\large\imprimirdata}
    \imprimirlocal, \imprimirDataDefesa.
    \vspace*{1cm}
  \end{center}
  }
  
% ----------------------------------------------------------
%\newcommand{\Errata}{%\color{blue}
%Elemento opcional da \textcite[4.2.1.2]{NBR14724:2011}. Exemplo:
%}

% ----------------------------------------------------------
\newcommand{\EpigrafeTexto}{%\color{blue}
\textit{"Qualquer árvore que queira tocar os céus\\
precisa ter raízes tão profundas\\
a ponto de tocar os infernos."\\
		Carl Gustav Jung}
}

% ----------------------------------------------------------
\newcommand{\ResumoTexto}{%\color{blue}
Dividido em duas partes, esse trabalho teve por objetivo explorar o problema inverso (PI) e suas aplicações voltadas para a modelagem, a validação e a linearização de amplificadores de potência (PAs), utilizando-se a modelagem computacional através de redes neurais artificiais como método para a reprodução do comportamento destes. Na primeira parte deste trabalho é analisado o problema inverso aplicado aos modelos de PAs, partindo-se do pressuposto de que construção de uma função inversa ao PA possui uma complexidade que pode ser relacionada ao quão não linear é sua é sua curva, sendo iniciada a construção de um arcabouço técnico de análises e um estudo de caso destas análises. O parâmetro analisado na primeira parte deste trabalho foi a distribuição dos resultados da resolução do problema inverso para o modelo direto do PA entre três categorias que surgiram naturalmente durante as análises exploratórias: convergência para o valor esperado, convergência para um valor não esperado e não-convergência. A segunda parte deste trabalho teve como foco o uso do problema inverso para a validação da pré-inversa de modelos de PAs sem o uso de novas medições, visto que o treinamento em cascata só garante a validação da pós-inversa do modelo e se torna necessário uma validação física desta em seu papel como pré-inversa.
}

\newcommand{\PalavraschaveTexto}{%\color{blue}
Problema inverso. Modelagem computacional. Rede neural artificial. Amplificador de potência. Linearização.}

% ----------------------------------------------------------
\newcommand{\AbstractTexto}{%\color{blue}
%This is the english abstract.
}
% ---
\newcommand{\KeywordsTexto}{%\color{blue}
%latex. abntex. text editoration.
}

% ----------------------------------------------------------
\newcommand{\Resume}
{%\color{blue}
%Il s'agit d'un résumé en français.
} 
% ---
\newcommand{\Motscles}
{%\color{blue}
 %latex. abntex. publication de textes.
}

% ----------------------------------------------------------
\newcommand{\Resumen}
{%\color{blue}
%Este es el resumen en español.
}
% ---
\newcommand{\Palabrasclave}
{%\color{blue}
%latex. abntex. publicación de textos.
}

% ----------------------------------------------------------
\newcommand{\AgradecimentosTexto}{%\color{blue}
Agradeço antes de tudo a minha família, em especial ao meu tio Reinaldo e a sua esposa Simone, que me apoiaram desde o começo dessa jornada de formas incomensuráveis.

Ao meu orientador, o Prof. Dr. Eduardo Gonçalves de Lima, que me orientou durante três anos de iniciação científica, na minha jornada dentro do curso de engenharia elétrica e que continua a me orientar no desenvolvimento deste trabalho.

Aos meus amigos e colegas de faculdade, pelo tempo compartilhado e apoio nas mais diversas atividades.

Aos professores com quem tive a sorte de compartilhar o espaço dentro da universidade e que muito me ensinaram durante a minha passagem por esse curso.

Ao GICS pelas oportunidades de pesquisa na iniciação científica cujo resultado final é esse trabalho de conclusão de curso.

Ao Departamento de Engenharia Elétrica e a Universidade Federal do Paraná, cuja estrutura possibilitou do inicio até o fim o meu aprendizado.

}

% ----------------------------------------------------------
\newcommand{\DedicatoriaTexto}{%\color{blue}
\textit{Esse trabalho é dedicado ao meu avô Thadeu (in memorian), o qual foi e continua sendo meu grande exemplo.}
	}

