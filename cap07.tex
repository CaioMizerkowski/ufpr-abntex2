\chapter{Conclusão} \label{cha:conclusao}
O problema inverso se mostra através dos resultados presentes uma ferramenta promissora para a análise do comportamento de modelos computacionais de PAs e de suas inversas. Existe uma grande região para se aprofundar nestas análises, sendo esse estudo de caso somente uma pequena amostra das possibilidades, que já é capaz de auxiliar a delimitar as regiões mais estáveis em um modelo a partir da aplicação de diferentes amplitudes como parâmetro inicial no processo de resolução do problema inverso.

Para evitar a necessidade de realizar uma nova medição do PA toda vez que um modelo diferente de DPD é treinado, esse trabalho propôs a solução de um problema inverso para modificar a entrada do DPD. Os estudos de caso analisados neste trabalho, que adotam um modelo de DPD baseado no TLP {para linearizar quatro diferentes PAs,} cada qual com características próprias de memória e de não-linearidade, mostraram um NMSE entre a medição da saída do PA e a entrada gerada do DPD de -41 dB para um PA Si LDMOS classe AB, de -39,3 dB para um PA GaN HEMT classe AB, de -39,7 dB para um PA GaN HEMT classe AB estimulado por duas portadoras e um de -36,9 dB para um PA GaN HEMT Doherty. {Uma comparação complementar, a qual utilizou um modelo polinomial de DPD, apontou, por meio do NMSE de -44,6 dB para a validação tradicional e de -38,9 dB para a validação proposta, que a metodologia proposta apresenta um resultado conservador em relação ao tradicional}. O conjunto de resultados indicou que o uso da metodologia proposta, usando somente as medições do PA coletadas antes do treinamento do DPD, pode ser uma alternativa para a validação de diferentes modelos de DPD, cujas características apresentem diferentes necessidades de memória e não-linearidades.
