\chapter{Conclusão} \label{cha:conclusao}
O primeiro estudo de caso mostra que o problema inverso é uma ferramenta promissora para a análise do comportamento de modelos computacionais de amplificadores de potência e de suas inversas, existindo uma grande região para se aprofundar tanto nos modelos utilizados para a modelagem computacional destes componentes como nas possibilidades de análises a serem realizadas. O resultado principal extraído desse estudo de caso foi mostrar que a alteração do valor inicial, na resolução do problema inverso, afeta significativamente a solução encontrada. A quantidade média de soluções corretas obtidas variou entre 94,92\% para a amplitude de 0,5 e 18,15\% para a amplitude de 2,5. Futuramente, análises mais profundas podem ser realizadas de forma a entender como estes valores variam em relação a diferentes amplificadores de potência e para diferentes valores iniciais usados. 

O segundo estudo de caso amplia o uso do problema inverso para uma aplicação mais direta, que é a validação de pré-distorcedores digitais através da solução de um problema inverso. As análises feitas neste estudo, que adotam um modelo de DPD baseado no MLP para linearizar quatro diferentes PAs, cada qual com características próprias de memória e de não-linearidade, mostraram um NMSE entre a medição da saída do PA e a entrada gerada do DPD de -41 dB para um PA Si LDMOS classe AB, de -39,3 dB para um PA GaN HEMT classe AB, de -39,7 dB para um PA GaN HEMT classe AB estimulado por duas portadoras e um de -36,9 dB para um PA GaN HEMT Doherty. Uma comparação complementar, a qual utilizou um modelo polinomial de DPD, apontou, por meio do NMSE de -44,6 dB para a validação tradicional e de -38,9 dB para a validação proposta, que a metodologia proposta apresenta um resultado conservador em relação ao tradicional. O conjunto de resultados indicou que o uso da metodologia, usando somente as medições do PA coletadas antes do treinamento do DPD, pode ser uma alternativa para a validação de diferentes modelos de DPD, cujas características apresentem diferentes necessidades de memória e não-linearidades.

Algumas entre as propostas para possíveis continuações deste trabalho são: a exploração mais profunda dos resultados obtidos através da utilização de um maior número de amplificadores de potência e da variação nos hiper parâmetros, a utilização de arquiteturas baseadas em redes neurais recorrentes (RNN) para a representação dos amplificadores de potência e a validação da metodologia proposta no segundo estudo de caso por meio de medidas físicas.
