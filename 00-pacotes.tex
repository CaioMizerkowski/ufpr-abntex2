% Pacotes básicos 
% ----------------------------------------------------------
%\usepackage{lmodern}			% Usa a fonte Latin Modern			
\usepackage[T1]{fontenc}		% Selecao de codigos de fonte.
\usepackage[utf8]{inputenc}		% Codificacao do documento (conversão automática dos acentos)
\usepackage{lastpage}			% Usado pela Ficha catalográfica
\usepackage{indentfirst}		% Indenta o primeiro parágrafo de cada seção.
\usepackage{color}		    	% Controle das cores
\usepackage{graphicx}			% Inclusão de gráficos
\usepackage{microtype} 			% para melhorias de justificação
\usepackage{ifthen}		    	% para montar condicionais
\usepackage[brazil]{babel}		% para utilizar termos em portugues
\usepackage[final]{pdfpages}    % para incluir páginas de arquivos pdf
\usepackage{lipsum}				% para geração de dummy text
\usepackage{csquotes}

%\usepackage[style=long]{glossaries}
%\usepackage{abntex2glossaries}

% permite representar o cancelamento de termos em texto ou equacoes
\usepackage{cancel} 		
% cores extendidas
\usepackage{xcolor} 		
% gera diagramas a partir de listas
\usepackage{smartdiagram}   
% Para a figura ficar na posição correta
\usepackage{float} 		    
% supporte para fontes da Text Companion 
\usepackage{textcomp} 		
% uso de longtable
\usepackage{longtable}		
% simbolos matematicos
\usepackage{amsmath}	
% páginas em paisagem
\usepackage{lscape}
% mescla de colunas em tabelas
\usepackage{multicol}
% mescla de linhas em tabelas
\usepackage{multirow}
% criação do indice de quadros
\usepackage{newfloat} 
% configura legenda 
\usepackage{caption} 		
%[format=plain]
	%\renewcommand\caption[1]{%
    \captionsetup{font=small}	% tamanho da fonte 10pt
    %,format=hang
 	% \caption{#1}}
	%\captionsetup{width=0.8\textwidth}

% Pacotes de citações BibLaTeX
% ----------------------------------------------------------
\usepackage[style=abnt,backref=true,repeatfields=true, backend=biber,citecounter=true,backrefstyle=three]{biblatex}


\DefineBibliographyStrings{brazil}{%
 backrefpage = {Citado \arabic{citecounter} vez na página},% originally "cited on page"
 backrefpages = {Citado \arabic{citecounter} vezes nas páginas},% originally "cited on pages"
}

% ----------------------------------------------------------

% alterando o aspecto da cor azul
\definecolor{blue}{RGB}{55,10,249}

% ----------------------------------------------------------
\PrepareListOf{quadro}{%
\renewcommand{\cftfigpresnum}{QUADRO~}}

\DeclareFloatingEnvironment[
fileext=loq,
listname={\textbf{LISTA DE QUADROS}},
name=Quadro,
%placement=p,
within= none, % numeracao continua
%within=section, % numeracao reinicia em cada seccao
%chapterlistsgaps=off
]{quadro}

\newlistentry{quadro}{loq}{0}


% Customize ‘List of Diagrams’
\PrepareListOf{quadro}{%
\renewcommand{\cftquadropresnum}{\normalsize{QUADRO}~}
\setlength{\cftquadronumwidth}{3.2cm}
%\renewcommand{\cftquadroname}{\quadroname\space} 
\renewcommand*{\cftquadroaftersnum}{\hfill--\hfill}
}

\makeatletter
%% we define a helper macro for adjusting lists of new floats to
%% accept a * behind them for not being shown in the TOC, like
%% the other list printing commands in memoir
\newcommand{\AdjustForMemoir}[1]{%
  \csletcs{kept@listof#1}{listof#1}%
  \csdef{listof#1}{%
    \@ifstar
     {\csappto{newfloat@listof#1@hook}{\append@star}%
      \csuse{kept@listof#1}}%
     {\csuse{kept@listof#1}}%
  }
}
\def\append@star#1{#1*}
\makeatother


\AdjustForMemoir{quadro} % prepare `\listofdirfigures` so it accepts a *

\makeatletter
\let\oldcontentsline\contentsline
\def\contentsline#1#2{%
    \expandafter\ifx\csname l@#1\endcsname\l@section
	\expandafter\@firstoftwo
	\else
	\expandafter\@secondoftwo
	\fi
	{%
		\oldcontentsline{#1}{\MakeTextUppercase{#2}}%
	}{%
	\normalsize %ajusta tamanho da fonte na lista
	\oldcontentsline{#1}{#2}%
}%
}
\makeatother

% Ajusta indentação de Referencias no ToC
% ----------------------------------------------------------
\defbibheading{bay}[\bibname]{%
  \chapter*{#1}%
  \markboth{#1}{#1}%
  \addcontentsline{toc}{chapter}
  {\protect\numberline{}\bibname}
}

\makeatletter
\pretocmd{\chapter}{\addtocontents{toc}{\protect\addvspace{-5\p@}}}{}{}
\pretocmd{\section}{\addtocontents{toc}{\protect\addvspace{2\p@}}}{}{}
\makeatother

% Para retirar os símbolos <> NBR6023:2018
\@ifpackageloaded{url}{%
\def\UrlLeft{}
\def\UrlRight{}}

% Para retirar os símbolos <> da URL  
\DeclareFieldFormat{illustrated}{\addspace #1\isdot}%
%\DeclareFieldFormat{url}{\bibstring{urlform}\addcolon\addspace<\url{#1}>}%
%\DeclareFieldFormat{url}{\bibstring{urlfrom}\addcolon\addspace<\url{#1}>}%
\DeclareFieldFormat{url}{\bibstring{}\addcolon\addspace \url{#1}}% remove <> em urls de acordo com abnt-6023:2018	

% Ajustar o espaço para a formatação da data
\DeclareFieldFormat{urldate}{\bibstring{urlseen}\addcolon\addspace #1}%

\DeclareFieldFormat*{note}{\addspace #1}%

% Para ajustar o tamanho da fonte do número da primeira página do capítulo
% comando utilizado na parte textual 
\makepagestyle{chapfirst}% Just for the first page of a chapter
\makeoddhead{chapfirst}{}{}{\normalsize{\thepage}}
