% ---
% Pacotes básicos 
% ---
\usepackage{lmodern}			% Usa a fonte Latin Modern			
\usepackage[T1]{fontenc}		% Selecao de codigos de fonte.
\usepackage[utf8]{inputenc}		% Codificacao do documento (conversão automática dos acentos)
\usepackage{lastpage}			% Usado pela Ficha catalográfica
\usepackage{indentfirst}		% Indenta o primeiro parágrafo de cada seção.
\usepackage{color}			% Controle das cores
\usepackage{graphicx}			% Inclusão de gráficos
\usepackage{microtype} 			% para melhorias de justificação
\usepackage{ifthen}			% para montar condicionais
\usepackage[brazil]{babel}		% para utilizar termos em portugues
% ---
% \usepackage{tocloft}		
\newcommand{\tocfont}{\normalsize}	% define tamanho de fonte para sumario como normal	
% ---
% Pacotes adicionais, usados apenas no âmbito do Modelo Canônico do abnteX2
% ---
\usepackage{lipsum}				% para geração de dummy text
% ---

\usepackage[style=long]{glossaries}


%\usepackage{abntex2glossaries}
% ---
% Pacotes de citações
% ---
\usepackage[brazilian,hyperpageref]{backref}	 % Paginas com as citações na bibl
\usepackage[alf,
	    %abnt-emphasize=em,  %\emph opção padrão.
            abnt-emphasize=bf  %\textbf
            ]{abntex2cite}	% Citações padrão ABNT

% --- 
% CONFIGURAÇÕES DE PACOTES
% --- 

% ---
% Configurações do pacote backref
% Usado sem a opção hyperpageref de backref
\renewcommand{\backrefpagesname}{Citado na(s) página(s):~}
% Texto padrão antes do número das páginas
\renewcommand{\backref}{}
% Define os textos da citação
\renewcommand*{\backrefalt}[4]{
	\ifcase #1 %
		Nenhuma citação no texto.%
	\or
		Citado na página #2.%
	\else
		Citado #1 vezes nas páginas #2.%
	\fi}%
% ---


% ---
% Configurações de aparência do PDF final

% alterando o aspecto da cor azul
\definecolor{blue}{RGB}{41,5,195}

% informações do PDF
\makeatletter
\hypersetup{
     	%pagebackref=true,
		pdftitle={\@title}, 
		pdfauthor={\@author},
    		pdfsubject={\imprimirpreambulo},
		pdfcreator={LaTeX with abnTeX2},
		pdfkeywords={abnt}{latex}{abntex}{abntex2}{trabalho acadêmico}, 
		colorlinks=true,       		% false: boxed links; true: colored links
    		linkcolor=blue,          	% color of internal links
    		citecolor=blue,        		% color of links to bibliography
    		filecolor=magenta,     		% color of file links
		urlcolor=blue,
		bookmarksdepth=4
}
\makeatother
% --- 


% ---
% compila o indice
% ---
\makeindex
% ---

\usepackage{cancel} 		% permite representar o cancelamento de termos em texto ou equacoes
\usepackage{xcolor} 		% cores extendidas
\usepackage{smartdiagram}       % gera diagramas a partir de listas

\usepackage{float} 		% Para a figura ficar na posição correta
\usepackage{textcomp} 		% supporte para fontes da Text Companion 
\usepackage{longtable}		% uso de longtable
\usepackage{amsmath}		% simbolos matematicos

\usepackage{caption} 		% configura legenda 
				%\renewcommand\caption[1]{%
\captionsetup{font=small}	% tamanho da fonte 10pt
 				% \caption{#1}}
				%\captionsetup{width=0.8\textwidth}

\usepackage{lscape}
\usepackage{multicol}
\usepackage{multirow}


\usepackage{trivfloat}
%\trivfloat{quadro}
%\floatstyle{plaintop}
%\restylefloat{quadro}

%%%%%%%%%%%%%%%%%%%%%%%%%%%%%
%Criação do indice de quadros
 \usepackage{newfloat}

\PrepareListOf{quadro}{%
\renewcommand{\cftfigpresnum}{Quadro~}}

\DeclareFloatingEnvironment[
fileext=loq,
listname={\textbf{LISTA DE QUADROS}},
name=Quadro,
%placement=p,
within= none, % numeracao continua
%within=section, % numeracao reinicia em cada seccao
%chapterlistsgaps=off
]{quadro}

% Novo list of (listings) para QUADROS

%\renewcommand{\quadroname}{QUADRO}
%\newcommand{\listofquadrosname}{LISTA DE QUADROS}

%\newfloat[chapter]{quadro}{loq}{\quadroname}
%\newlistof{listofquadros}{loq}{\listofquadrosname}
\newlistentry{quadro}{loq}{0}

% configurações para atender às regras da ABNT
%\counterwithout{quadro}{chapter}

% Customize ‘List of Diagrams’
\PrepareListOf{quadro}{%
\renewcommand{\cftquadropresnum}{\normalsize{QUADRO}~}
\setlength{\cftquadronumwidth}{3.2cm}
%\renewcommand{\cftquadroname}{\quadroname\space} 
\renewcommand*{\cftquadroaftersnum}{\hfill--\hfill}
}

\makeatletter
%% we define a helper macro for adjusting lists of new floats to
%% accept a * behind them for not being shown in the TOC, like
%% the other list printing commands in memoir
\newcommand{\AdjustForMemoir}[1]{%
  \csletcs{kept@listof#1}{listof#1}%
  \csdef{listof#1}{%
    \@ifstar
     {\csappto{newfloat@listof#1@hook}{\append@star}%
      \csuse{kept@listof#1}}%
     {\csuse{kept@listof#1}}%
  }
}
\def\append@star#1{#1*}
\makeatother

\AdjustForMemoir{quadro} % prepare `\listofdirfigures` so it accepts a *

\makeatletter
\let\oldcontentsline\contentsline
\def\contentsline#1#2{%
    \expandafter\ifx\csname l@#1\endcsname\l@section
	\expandafter\@firstoftwo
	\else
	\expandafter\@secondoftwo
	\fi
	{%
		\oldcontentsline{#1}{\MakeTextUppercase{#2}}%
	}{%
	\normalsize %ajusta tamanho da fonte na lista
	\oldcontentsline{#1}{#2}%
}%
}
\makeatother

%%%%%%%%%%%%%%%%%%%%%%%%%%%%%%%%%%%%%%%%%%%%%%%%%
% citacao de acordo com a norma da UFPR 2015-agosto
\renewenvironment{quotation}
    {\vspace{.5\baselineskip}
	\list{}{\listparindent 0em%
		\SingleSpacing
		\itemindent    
		\listparindent
		\rightmargin= 0mm
        \leftmargin= 45mm
        \parsep=0mm        }%\z@ \@plus\p@}%
        \item\relax}
        {
        \vspace{.5\baselineskip}
        \endlist
        }
        
\newenvironment{citacaodireta}
{\begin{quotation}\footnotesize}
{\end{quotation}}        

\newcommand{\sugest}[1]{\textcolor{red!40}{#1}}
